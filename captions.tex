% Fig. 1
\caption{
(a) Heatmap of XRD dataset $X_{iq}$ corresponding to a temperature scan
of XRD patterns for xxxxx system. The vertical index $i$ and horizontal
index $q$ index temperature and momentum transfer, respectively.
}

% Fig. 2: summary figure showing signal separation and similarity
% squares for T progression dataset.

% Fig. 3: Example of the effect of background subtraction and noise
% estimation on peak fitting for one BBA block. This figure needs to be
% reformatted (font size, layout)

% Fig. 4: feature map and peak shift-corrected features

% Fig. 5: comparison of similarity squares for raw XRD, background
% subtracted XRD, and peak shift-corrected XRD

% Fig. 6: 3d raw data, 2d (cross section) feature map, and 3d feature
% 'plates' for Suchi's dataset

% Fig. 7: Summary plot for one slice of Suchi's dataset (same format as Fig. 2)

% Fig. 8: 
\caption{
(a) Optimal clustering for a set of XRD patterns demonstrating a
temperature progression in xxxx system. The optimal number of clusters
is determined by clustering stability over an ensemble of XRD patterns
generated according to the simple noise model described in section xxxx.

(b) Ensemble probability of the most probable cluster configuration
for agglomerative clustering with Ward linkage criterion, for the same
dataset. Horizontal axis varies the number of clusters.
}
